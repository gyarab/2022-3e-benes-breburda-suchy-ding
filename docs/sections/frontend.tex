\section{Frontend}

Celý frontend naší aplikace je postavený ve Vue.js, moderním
JavaScriptovém frameworku určeného k tvorbě uživatelského rozhraní.
Zaměřuje se na deklarativní renderování a složení komponent.
Pro komunikaci frontendu s backendem pracujeme s REST API, který
pro přenos dat využívá základních protokolů a technologií, jako je
protokol HTTP, přičemž těla požadavků a odpovědí jsou ve formátu JSON.

\subsection{Design UI}

Vzhled uživatelského rozhraní je samozřejmě jednou z nejdůležitějších
vlastností stránky sociální sítě. Lišit se může v závislosti například
na cílové skupině uživatelů, či zaměření sítě. Nicméně, ty nejdůležitější 
obecné principy jsme museli dodržovat a brát v úvahu i při našem návrhu.
Design musí být konsistentní, přehledný a jednoduchý. Když se se stránkou
uživatel setká poprvé, nemusí ho nijak zvlášť ohromit, stačí, aby se v ní
byl hned schopný zorientovat a bezmyšlenkovitě s ní interagovat. Hlavní
funkce a navigace by měly být intuitivní, viditelné a snadno dostupné. 

Je nesmírně důležité, aby byl v celé aplikaci zachován jednotný design 
daných prvků designu (jako jsou např. tlačítka, ikony, či formulářové pole)
a aby si tyto prvky se sebou navzájem dobře vizuálně vycházeli. Proto jsme
v průběhu projektu vybírali jen z úzkého výběru barev, kterými jsou hlavně 
odstíny tmavší šedi, které se navzájem doplňujou, nikoli dráždí. Tyto šedé
tóny konstrastuje pestrá fialová, kterou využíváme k zbarvení a odlišení 
některých funkcí stránky. Křiklavý rozpor šedého pozadí a fialových
detailů uživateli poukazuje na důležitost těchto funkcí. Touto fialovou se
rozhodně šetří, zesiluje tím právě dojem její rarity a důsledně i hodnoty.
Právě touto barvou se hlavně chlubí všudypřítomné tlačítko "New Post" 
(Nový příspěvek), které i svou velikostí je pro uživatelé naproto
nepřehlédnutelné.

Vedle barev prkvů je ale samozřejmě důležitý i jejich tvar a viditelný obsah.
Chtěli jsme, aby stránka působila přátelsky a přívětivě. Všechna tlačítka,
příspěvky a inputové pole jsou zaoblovány, zdají se tak sympatičtější. Na 
ostrou hranu prakticky napříč celým programem nenarazíte, tento trend ctí
i font hlavního loga, se kterým se nový uživatel setkává například hned u
vytváření vlastního účtu. I ikony, se kterými jsme se rozhodli pracovat ze
sady "heroicons" působí na oko příjemně.

\subsection{Router}

Vue router je oficiální router pro Vue.js. Důvodem, proč jsme se rozhodli
pracovat s Vue router je, že nám umožňuje přepínat mezi jednotlivými stránkami
bez nutnosti načítat celou stránku znovu, což je výhodné pro uživatele, protože
se mu stránka bude načítat rychleji, což zlepší celkový dojem z aplikace.

\subsection{Komponenty}

Komponenty jsou základní stavební jednotkou Vue.js aplikace. Jsou to vlastně
opakující se bloky kódu, které můžeme použít k vytvoření uživatelského rozhraní.
Komponenty jsou znovupoužitelné a můžeme je použít k vytvoření stejných bloků
kódu v aplikaci. Komponenty jsou velmi užitečné pro vytváření opakujících se
elementů, jako jsou například příspěvky, komentáře, uživatelské profily, atd.
Komponenty jsou také velmi užitečné pro udržování kódu, protože můžeme použít
stejnou komponentu v různých částech aplikace.

Komponenty jsou základní stavební jednotkou Vue.js aplikace. Využíváme je
k vytvoření uživatelského rozhraní. Komponenty jsou znovupoužitelné a díky tomu
nemusíme po každé vytvářet stejné bloky kódu, díky čemuž šetříme čas a zjednodušujeme
si práci. Komponenty jsou také velmi užitečné pro udržování kódu, protože můžeme
použít stejnou komponentu v různých částech aplikace.

Importujeme je do souboru, kde
je můžeme zavolat. Komponenty jsou velmi užitečné pro vytváření opakujících se
elementů, jako jsou například příspěvky, komentáře, nebo nastavení uživatelského
profilu. 
