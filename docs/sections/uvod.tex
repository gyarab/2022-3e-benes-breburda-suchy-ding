\section{Úvod}
Ding je sociální síť zaměřená na nahrávky zvuků. Mluvené slovo je na síti zakázáno
a automaticky blokováno. Uživatelé se mohou navzájem poslouchat a na své příspěvky
odpovídat.

\subsection{Původní znění zadání}
Cílem by bylo vytvořit sociální síť, na kterou by se nedalo přidat nic jiného
než zvukové soubory, které by nesměly obsahovat mluvené slova (nebo by množství
bylo omezené). Mluvená slova by se automaticky filtrovala. Aplikace by byla
webová stránka.

UI by bylo ve stylu Twitteru, ale upravené pro audio formát. Uživatel by mohl
nahrávat audio hned na webové stránce, nebo nahrát už vytvořený soubor. Měl by
také feed audia od dalších uživatelů.

Později bychom mohli pomocí elektronu udělat android / IOS aplikaci. Primárně
by ale aplikace žila na webu. 

\subsection{Aplikace}
Sociální sítě slouží jako platformy pro sdílení informací, zábavu, komunikaci
a sociální interakce online. V dnešní době jsme nimi obklopeni naprosto ze všech
stran a každý dobře víme jak taková platforma vypadá, jak přibližně funguje a co
od ni můžeme očekávat. Základní funkčnosti spolu většina těchto sítí sdílí:
přidávání příspěvků, sledování uživatelů, komentáře atd. Naši sociální síť 
od těch ostatních rozlišuje primárně obsah jednotlivých příspěvků. Audio.

Ačkoli byly sociální sítě vynalezeny hlavně pro účel rychleji a efektivněji
spojovat a propojovat jednotlivé uživatele, dnes už nesou zodpovědnost za 
mnohem více. Díky obrovskému počtu denodenních uživatelů a zájem obyčejných
lidí se staly tyto prostory centrem reklam, kde každá firma a každá společnost
soupeří o čas a pozornost svých uživatelů. To dělají nejrůznějšími způsoby.

Prostost audia a zvukových příspěvků je v dnešní době příjemné zjednodušení.
Příspěvky jsou v naší aplikaci bez popisků a náhledových obrázků, které by
vás chtěli vtáhnout k sobě. Uživatel nikdy neví co přímo očekávat.
Každý příspěvek jinak obsahuje viditelné informace ohledně autorského
uživatele (profilová fotka a jméno), možnost příspěvku přidat Ding (ekvivalent
klasického Like, se kterými se setkáme kdekoliv jinde), či ho okomentovat.
Dále lze zobrazit komentáře od ostatních uživatelů, případně jakéhokoliv z nich
navštívit na jeho profilu a poslechnout ji všechny jejich příspěvky, nebo si 
příspěvek uložit do složky Saved Posts, kde ho v seznamu všech vašich ostatních 
uložených příspěvků kdykoliv zase naleznete.

\newpage
